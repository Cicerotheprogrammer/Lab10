\documentclass[a4paper]{article}
\usepackage{labreport}

\begin{document}

\section{Objective}
The objective is to measure the time constant of an RC circuit in order to verify the calculated values \cite{UNCC-ECE-Dept:2023}.
\section{Equipment Used}

\begin{itemize}
    \item Digital Multimeter
    \item DC Power Supply
    \item Resistor: $20k\Omega$
    \item Capacitor: $2,200 \mu$F
    \item Alligator (Clips) Jumper
\end{itemize}

\section{Experiment Setup}

\begin{enumerate}
    \item Construct the circuit in Figure 10-1\cite{UNCC-ECE-Dept:2023}.
    \begin{figure}[H]\label{fig10-1}
        \begin{center}
            \includegraphics[width = 7 cm]{fig10-1}\\
            \small\textbf{Figure 10-1 Series RC Circuit for Experimental Setup \cite{UNCC-ECE-Dept:2023}}
        \end{center}
    \end{figure} 
    \item Measure the initial current \cite{UNCC-ECE-Dept:2023}.
    \item Calculate the values of $\tau$ and 5$\tau$ in which $\tau$ is equal to $R_{eq}$ times $C_{eq}$ \cite{UNCC-ECE-Dept:2023}.
    \item Collect data by timing the measurements of the current to ever 15 seconds and in order to measure the current remove the alligator clip from the circuit \cite{UNCC-ECE-Dept:2023}.
    \item Take measurements every 15 seconds until 5$\tau$ and possibly further \cite{UNCC-ECE-Dept:2023}.
    \item Repeat the previous 2 steps for trial 2 \cite{UNCC-ECE-Dept:2023}.
    \item Take the average values of the two trials at each measured time \cite{UNCC-ECE-Dept:2023}.
    \item Use the average current to calculate the voltage across the resistor \cite{UNCC-ECE-Dept:2023}.
    \item Solve the voltage across the capacitor, $V_{c}$, which can be solved fro with the voltage accross the resistor $V_{c}$ = ($V_{s}$ - $V_{R}$) \cite{UNCC-ECE-Dept:2023}.  
\end{enumerate}


\section{Results}
$\tau$ = 44 seconds  Initial Current = 1.745 mA  \\
5$\tau$ =  220 seconds \\

\begin{center}
    \small\textbf{Table 10-1: Data Table for RC Time Constant \cite{UNCC-ECE-Dept:2023}}\\
    \begin{tabular}{|p{2 cm}|p{2 cm}|p{2 cm}|p {2 cm}|p {2 cm}|p{2 cm}|}
        \hline
        Time (min:sec) & \multicolumn{3}{|c|}{Current (mA)} & Resistor Voltage (V) & Capacitor Voltage (V) \\
        \hline
        & Trial 1 & Trial 2 & Average & & \\
        \hline
        0:00 & 1.745 mA & 1.745 mA & 1.745 mA & 34.9 V  & 0 V \\
        \hline
        0:15 & 1.242 mA & 1.24 mA & 1.241 mA & 24.82 V & 10.08 V \\
        \hline
        0:30 & .901 mA & .900 mA & .9005 mA & 18.01 V &  16.86 V \\
        \hline
        0:45 $\tau$ & .662 mA & .666 mA & .664 mA & 13.28 V & 21.62 V \\
        \hline
        1:00 & .490 mA & .500 mA & .495 mA & 9.9 V & 25 V \\
        \hline
        1:15 & .365 mA & .373 mA & .369 mA & 7.38 V & 27.52 V \\
        \hline
        1:30 & .277 mA & .274 mA & .2755 mA & 5.51 V & 29.39 V \\
        \hline
        1:45 & .212 mA & .207 mA & .2095 mA & 4.19 V & 30.71 V \\
        \hline
        2:00 & .165 mA & .158 mA & .1615 mA & 3.23 V & 31.67 V \\
        \hline
        2:15 & .125 mA & .112 mA & .1185 mA & 2.37 V & 32.53 V \\
        \hline
        2:30 & .100 mA & .092 mA & .096 mA & 1.92 V & 32.98 V \\
        \hline
        2:45 & .076 mA & .072 mA & .074 mA & 1.48 V & 33.42 V \\
        \hline
        3:00 & .062 mA & .057 mA & .0595 mA & 1.19 V & 33.71 V \\
        \hline
        3:15 & .050 mA & .045 mA & .0475 mA & .95 V & 33.95 V \\
        \hline
        3:30 & .041 mA & .036 mA & .0385 mA & .77 V & 34.13 V \\
        \hline
        3:45 & .034 mA & .029 mA & .0315 mA & .63 V & 34.27 V \\
        \hline
        4:00 & .028 mA & .024 mA & .026 mA & .52 V & 34.38 V \\
        \hline
        4:15 & .023 mA & .020 mA & .0215 mA & .43 V & 34.47 V \\
        \hline
        4:30 & .020 mA & .016 mA & .018 mA & .36 V & 34.54 V \\
        \hline
        4:45 5$\tau$ & .017 mA & .014 mA & .0155 mA & .31 V & 34.59 V \\
        \hline
        5:00 & .015 mA & .012 mA & .0135 mA & .27 V & 34.63 V \\
        \hline
        5:15 & .013 mA & .010 mA & .0115 mA & .23 V & 34.67 V \\
        \hline
        5:30 & .011 mA & .009 mA & .010 mA & .2 V & 34.7 V \\
        \hline
        5:45 & .010 mA & .008 mA & .009 mA & .18 V & 34.72 V \\
        \hline
        6:00 & .009 mA & .007 mA & .008 mA & .16 V & 34.74 V \\
        \hline
    \end{tabular}
\end{center}



\section{Conclusion}

As a capacitor is charged current will decrease until the capacitor is fully charged thus making an open circuit. Furthermore, the voltage will be collected into the capacitor as the voltage across the resistor goes down.

\bibliography{references}
\bibliographystyle{plain}



\end{document}